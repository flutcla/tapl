\documentclass[%
  unicode,
  dvipdfmx,
  mathserif,
  % notheorems,
  m, % vertical position of text: (t,m,b)
  % handout,
  aspectratio=169,
  12pt,
  % compress,
  % unknownkeysallowed
]{beamer}

\usepackage{minijs}
\usepackage{mathrsfs}
\usepackage{amsmath,amsfonts,amsthm,amssymb}
\usepackage{mathtools}
\usepackage{color}
\usepackage{graphicx}
\usepackage{tikz}
\usepackage{bm,bbm}
\usepackage{picture}
\usepackage{listings}
\lstset{
  basicstyle=\ttfamily,
  mathescape
}
\usepackage{url}
% https://www.logicmatters.net/resources/ndexamples/proofsty.html
\usepackage{proof}
% https://www.ccs.neu.edu/home/wand/csg264/latex/mathpartir/mathpartir.pdf
\usepackage{mathpartir}

\usepackage{ifthen}
\newboolean{isHandout}
\newcommand{\handout}[1]{\ifthenelse{\boolean{isHandout}}{#1}{}\note{#1}}
\newcommand{\slide}[1]{\ifthenelse{\boolean{isHandout}}{}{#1}}

\newcommand{\subs}[1]{\textsubscript{#1}}
\newenvironment{proposition}{\begin{fact}}{\end{fact}}
\renewcommand{\\}{\newline}

\newcommand*{\myit}[1]{\tikz[baseline=(T.base)]{%
  \node[inner sep=0pt,outer sep=0pt,xslant=0.25](T){#1}}}

\usepackage{stmaryrd}

\setboolean{isHandout}{false}
\usepackage{bxdpx-beamer}
\usepackage{pxjahyper}
\usepackage{fancybox}
\usefonttheme{professionalfonts}

% https://ftp.jaist.ac.jp/pub/CTAN/macros/latex/contrib/beamer-contrib/themes/metropolis/doc/metropolistheme.pdf
\usetheme{metropolis}
\metroset{
  sectionpage = progressbar,  % none, simple, progressbar
  subsectionpage = progressbar,  % none, simple, progressbar
  numbering = fraction,  % none, counter, fraction
  progressbar = none,  % none, head, frametitle, foot
  block = fill,  % transparent, fill .
}

\makeatletter
\def\beamer@framenotesbegin{% at beginning of slide
\usebeamercolor[fg]{normal text}
\gdef\beamer@noteitems{}%
\gdef\beamer@notes{}%
}
\makeatother

\definecolor{royalblue}{rgb}{0.255, 0.412, 0.882}
\setbeamercolor{frametitle}{fg=white, bg=royalblue}
\definecolor{aliceblue}{rgb}{0.94, 0.97, 1.0}
\setbeamercolor{normal text}{fg=black, bg=aliceblue}
\setbeamercolor{alerted text}{fg=black}
\setbeamercolor{example text}{fg=black}

% \usepackage{pgfpages}
% % \setbeameroption{hide notes} % スライドのみ作成
% % \setbeameroption{show only notes} % 発表ノートのみ作成
% \setbeameroption{show notes on second screen=right} % スライドと発表ノートを横並びに作成
% \setbeamertemplate{note page}{\pagecolor{gray!10}\insertnote}

\title{ TaPL Seminar }
\subtitle{ 3.1 - 3.4 }
\author{ Kosuke Kiuchi }
\date{March 20th, 2023}

\begin{document}
\maketitle
\uselanguage{japanese}
\languagepath{japanese}
\deftranslation[to=japanese]{Fact}{Proposition}

\section*{3 Untyped Arithmetic Expressions}

\subsection*{3.1 Introduction}

\begin{frame}[fragile]{Language}
% \centering
\begin{lstlisting}
t $\Coloneqq$ true
     false
     if t then t else t
     0
     succ t
     pred t
     iszero t
\end{lstlisting}
\handout{
Symbol \texttt{t} in the right-hand sides is called a \textit{metavariable}.\\
It is a place-holder for some particular term.\\
"Meta" is because it is a variable of the \textit{metalanguage}, not \textit{object language}.\\
}\slide{
  \texttt{t} : \textit{metavariable}
}
\note{
  扱う言語の項は図のように表される \\
  今の所、項と式は同義 \\
}
\end{frame}

\begin{frame}[fragile]{Program}
In the present language, program = term.
\begin{lstlisting}
  if false then 0 else 1;
$\blacktriangleright$ 1
  iszero (pred (succ 0));
$\blacktriangleright$ true
\end{lstlisting}
\handout{
The results of evaluations are either boolean constants or numbers.\\
Such terms are called \textit{values}.\\
Notice that the syntax can make terms like \texttt{succ true} and \texttt{if 0 then 0 else 0}.
}\slide{
Results are boolean constants or numbers, called \textit{values}.\\
Notice: \texttt{succ true}, \texttt{if 0 then 0 else 0}, \dots are allowed this time.
}
\end{frame}

\subsection*{3.2 Syntax}

\begin{frame}{Ways to define terms}
\begin{itemize}
  \item Inductively
  \item Rule Inferred
  \item Concretely
\end{itemize}
\end{frame}

\begin{frame}{Inductive Definition}
\begin{definition}[3.2.1 Terms, Inductively]
The set of \textit{terms} is the smallest set $\mathcal{T}$ such that
\begin{enumerate}
  \item $\{ \texttt{true, false, 0} \} \subseteq \mathcal{T}$;
  \item if $\texttt{t\subs{1}} \in \mathcal{T}$, then $\{ \texttt{succ t\subs{1}, pred t\subs{1}, iszero t\subs{1}} \} \subseteq \mathcal{T}$;
  \item if $\texttt{t\subs{1}} \in \mathcal{T}$, $\texttt{t\subs{2}} \in \mathcal{T}$, and $\texttt{t\subs{3}} \in \mathcal{T}$, then $\texttt{if t\subs{1} then t\subs{2} else t\subs{3}} \in \mathcal{T}$.
\end{enumerate}
\end{definition}
\note{
1 は T の基本的な項 \\
2,3 は複合的な式が T に属しているか判定する方法 \\
"smallest" は T に余分な項が含まれていないことを示す
}
\end{frame}

\begin{frame}{Definition by Inference Rules}
\begin{definition}[3.2.2 Terms, by Inference Rules]
The set of terms is defined by the following rules:
\begin{mathpar}
  \texttt{true} \in \mathcal{T} \and \texttt{false} \in \mathcal{T} \and \texttt{0} \in \mathcal{T} \\
  \infer{\texttt{succ t\subs{1}} \in \mathcal{T}}{\texttt{t\subs{1}} \in \mathcal{T}} \and
  \infer{\texttt{pred t\subs{1}} \in \mathcal{T}}{\texttt{t\subs{1}} \in \mathcal{T}} \and
  \infer{\texttt{iszero t\subs{1}} \in \mathcal{T}}{\texttt{t\subs{1}} \in \mathcal{T}} \\
  \infer{\texttt{if t\subs{1} then t\subs{2} else t\subs{3}} \in \mathcal{T}}{\texttt{t\subs{1}} \in \mathcal{T} & \texttt{t\subs{2}} \in \mathcal{T} & \texttt{t\subs{3}} \in \mathcal{T}}
\end{mathpar}
\end{definition}
Each rule is called \textit{inference rule}.
\handout{
Each rule is read, "If we have established the statements in the premise(s) listed above the line, then we may derive the conclusion below the line."
\begin{itemize}
\item Rules with no premises are often called \textit{axioms}.
\item The term \textit{inference rule} includes both axioms and rules with one or more premises.
\item Axioms are usually written with no bar.
\end{itemize}
}
\end{frame}

\begin{frame}{Concrete Definition}
\handout{
More "concrete" style that gives an explicit procedure for \textit{generating} the elements of $\mathcal{T}$.
}
\begin{definition}[3.2.3 Terms, Concretely]
For each natural number $i$, define a set $\mathcal{S}_i$ as follows:
\begin{alignat*}{2}
  \mathcal{S}_0 &= \emptyset \\
  \mathcal{S}_{i+1} &= & &\{ \texttt{true, false, 0} \}\\
  &&\cup &\{ \texttt{succ t\subs{1}, pred t\subs{1}, iszero t\subs{1}} \vert \texttt{t\subs{1}} \in \mathcal{S}_i \}\\
  &&\cup &\{ \texttt{if t\subs{1} then t\subs{2} else t\subs{3}} \vert \texttt{t\subs{1}, t\subs{2}, t\subs{3}} \in \mathcal{S}_i \}\\
\end{alignat*}
Finally, let
$\mathcal{S} = \bigcup_{i} \mathcal{S}_i$.
\end{definition}
\end{frame}

\begin{frame}{Ex. 3.2.4 [$\bigstar\bigstar$] How many elements does $\mathcal{S}_3$ have?}
\pause
\begin{alignat*}{2}
  \only<2>{\mathcal{S}_0 &= &&\emptyset \\}
  \only<2>{\therefore}|\mathcal{S}_0| &= &&0\\
  \only<2>{\mathcal{S}_{i+1} &= &&\{ \texttt{true, false, 0} \}\\
  &&\cup &\{ \texttt{succ t\subs{1}, pred t\subs{1}, iszero t\subs{1}} \vert \texttt{t\subs{1}} \in \mathcal{S}_i \}\\
  &&\cup &\{ \texttt{if t\subs{1} then t\subs{2} else t\subs{3}} \vert \texttt{t\subs{1}, t\subs{2}, t\subs{3}} \in \mathcal{S}_i \}\\}
  \only<2>{\therefore}|\mathcal{S}_{i+1}| &= &&3 + 3 \times |\mathcal{S}_i| + |\mathcal{S}_i|^3
\end{alignat*}
\pause
\begin{alignat*}{3}
  |\mathcal{S}_1| &= &&3 + 3 \times 0 + 0^3 && = 3\\
  |\mathcal{S}_2| &= &&3 + 3 \times 3 + 3^3 && = 39\\
  |\mathcal{S}_3| &= &&3 + 3 \times 39 + 39^3 && = 59439
\end{alignat*}
\end{frame}


\begin{frame}{Ex. 3.2.5 [$\bigstar\bigstar$] Show that $\mathcal{S}_i \subseteq \mathcal{S}_{i+1}$}
\pause
Prove inductively. Assume that $\texttt{t} \in \mathcal{S}_{i}$.\\
\begin{itemize}
  \item If \texttt{t} is either \texttt{true, false, 0}, obvious.
  \pause
  \item If \texttt{t} has the form \texttt{succ t\subs{1}}, $\mathtt{t\subs{1}} \in \mathcal{S}_{i-1}$ holds.
        From induction hypothesis, $\mathtt{t\subs{1}} \in \mathcal{S}_i$.\\
        Therefore $\texttt{succ t\subs{1}} \in \mathcal{S}_{i+1}$.
        The same holds for \texttt{pred} and \texttt{iszero}.
  \pause
  \item If \texttt{t} has the form \texttt{if t\subs{1} then t\subs{2} else t\subs{3}}, $\mathtt{t\subs{1}, t\subs{2}, t\subs{3}} \in \mathcal{S}_{i-1}$ holds.\\
        From induction hypothesis, $\mathtt{t\subs{1}, t\subs{2}, t\subs{3}} \in \mathcal{S}_i$.\\
        Therefore $\texttt{if t\subs{1} then t\subs{2} else t\subs{3}} \in \mathcal{S}_{i+1}$
\end{itemize}
\end{frame}

\begin{frame}{Two Views Define the Same Set}
\begin{proposition}[3.2.6]
  $\mathcal{T} = \mathcal{S}$
\end{proposition}
\begin{proof}
  Read p.28.
\end{proof}
\end{frame}

\subsection*{3.3 Induction on Terms}

\begin{frame}{Inductive}
If $\mathtt{t} \in \mathcal{T}$, then one of three things must be true:
\begin{enumerate}
  \item \texttt{t} is a constant.
  \item \texttt{t} has the form \texttt{succ t\subs{1}, pred t\subs{1}}, or \texttt{iszero t\subs{1}} for some \textit{smaller} term \texttt{t\subs{1}}.
  \item \texttt{t} has the form \texttt{if t\subs{1} then t\subs{2} else t\subs{3}} for some \textit{smaller} terms \texttt{t\subs{1}, t\subs{2} and t\subs{3}}
\end{enumerate}
We can
\begin{itemize}
  \item give \textit{inductive definitions} of functions.
  \item give \textit{inductive proofs} of properties of terms.
\end{itemize}
\end{frame}

\begin{frame}{Inductive Definitions of Consts(\texttt{t})}
\begin{definition}[3.3.1]
\begin{alignat*}{2}
  &Consts(\texttt{true}) &&= \{\texttt{true}\}\\
  &Consts(\texttt{false}) &&= \{\texttt{false}\}\\
  &Consts(\texttt{0}) &&= \{\texttt{0}\}\\
  &Consts(\texttt{succ t\subs{1}}) &&= Consts(\texttt{t\subs{1}})\\
  &Consts(\texttt{pred t\subs{1}}) &&= Consts(\texttt{t\subs{1}})\\
  &Consts(\texttt{iszero t\subs{1}}) &&= Consts(\texttt{t\subs{1}})\\
  &Consts(\texttt{if t\subs{1} then t\subs{2} else t\subs{3}}) &&= Consts(\texttt{t\subs{1}}) \cup Consts(\texttt{t\subs{2}}) \cup Consts(\texttt{t\subs{3}})
\end{alignat*}
\end{definition}
\end{frame}

\begin{frame}{Inductive Definition of size(\texttt{t})}
\begin{definition}[3.3.2 size(\texttt{t})]
\begin{alignat*}{2}
  &size(\texttt{true}) &&= 1\\
  &size(\texttt{false}) &&= 1\\
  &size(\texttt{0}) &&= 1\\
  &size(\texttt{succ t\subs{1}}) &&= size(\texttt{t\subs{1}}) + 1\\
  &size(\texttt{pred t\subs{1}}) &&= size(\texttt{t\subs{1}}) + 1\\
  &size(\texttt{iszero t\subs{1}}) &&= size(\texttt{t\subs{1}}) + 1\\
  &size(\texttt{if t\subs{1} then t\subs{2} else t\subs{3}}) &&= size(\texttt{t\subs{1}}) + size(\texttt{t\subs{2}}) + size(\texttt{t\subs{3}}) + 1
\end{alignat*}
\end{definition}
\handout{The size of \texttt{t} is the number of nodes in its abstract syntax tree.}
\end{frame}

\begin{frame}{Inductive Definition of depth(\texttt{t})}
\begin{definition}[3.3.2 depth(\texttt{t})]
\begin{alignat*}{2}
  &depth(\texttt{true}) &&= 1\\
  &depth(\texttt{false}) &&= 1\\
  &depth(\texttt{0}) &&= 1\\
  &depth(\texttt{succ t\subs{1}}) &&= depth(\texttt{t\subs{1}}) + 1\\
  &depth(\texttt{pred t\subs{1}}) &&= depth(\texttt{t\subs{1}}) + 1\\
  &depth(\texttt{iszero t\subs{1}}) &&= depth(\texttt{t\subs{1}}) + 1\\
  &depth(\texttt{if t\subs{1} then t\subs{2} else t\subs{3}}) &&= \max(depth(\texttt{t\subs{1}}), depth(\texttt{t\subs{2}}), depth(\texttt{t\subs{3}})) + 1
\end{alignat*}
\end{definition}
\handout{The depth of \texttt{t} is the smallest $i$ such that $\texttt{t} \in \mathcal{S}_i$.}
\end{frame}

\begin{frame}{Inductive Proof of a Simple Fact}
\begin{lemma}[3.3.3]
$|Consts(\texttt{t})| \le size(\texttt{t})$
\end{lemma}
\begin{proof}
\pause
\only<2>{
By induction on the depth of \texttt{t}.\\
Case: \texttt{t} is a constant\\
Immediate: $|Consts(\texttt{t})| = |\{\texttt{t}\}| = 1 = size(\texttt{t})$.\\
Case: \texttt{t} = \texttt{succ t\subs{1}, pred t\subs{1}} or \texttt{iszero t\subs{1}}\\
By the IH, $|Consts(\texttt{t\subs{1}})| \le size(\texttt{t\subs{1}})$.\\
$\therefore |Consts(\texttt{t})| = |Consts(\texttt{t\subs{1}})| \le size(\texttt{t\subs{1}}) < |Consts(\texttt{t})|$
}
\pause
\only<3>{
\small{
Case: \texttt{t} = \texttt{if t\subs{1} then t\subs{2} else t\subs{3}}\\
By the IH, $|Consts(\texttt{t\subs{1}})| \le size(\texttt{t\subs{1}})$, $|Consts(\texttt{t\subs{2}})| \le size(\texttt{t\subs{2}})$ and $|Consts(\texttt{t\subs{3}})| \le size(\texttt{t\subs{3}})$.
\begin{align*}
  \therefore |Consts(\texttt{t})| &= |Consts(\texttt{t\subs{1}}) \cup Consts(\texttt{t\subs{2}}) \cup Consts(\texttt{t\subs{3}})|\\
  &\le |Consts(\texttt{t\subs{1}})| + |Consts(\texttt{t\subs{2}})| + |Consts(\texttt{t\subs{3}})|\\
  &\le size(\texttt{t\subs{1}}) + size(\texttt{t\subs{2}}) + size(\texttt{t\subs{3}})\\
  &< size(\texttt{t}). \qedhere
\end{align*}
}
}
\end{proof}
\end{frame}

\begin{frame}{Thress Inductions on Terms}
\begin{theorem}[3.3.4 Principles of Induction on Terms]
  Induction on depth:\\
  \quad $\forall \texttt{s} \in \mathcal{T},
  (\forall \texttt{r}\in\mathcal{T} ~s.t.~ depth(\texttt{r}) < depth(\texttt{s}), P(\texttt{r}) \rightarrow P(\texttt{s}))
  \\\qquad\rightarrow \forall \texttt{s} \in \mathcal{T}, P(\texttt{s})$.\\
  Induction on size:\\
  \quad $\forall \texttt{s} \in \mathcal{T},
  (\forall \texttt{r}\in\mathcal{T} ~s.t.~ size(\texttt{r}) < size(\texttt{s}), P(\texttt{r}) \rightarrow P(\texttt{s}))
  \\\qquad\rightarrow \forall \texttt{s} \in \mathcal{T}, P(\texttt{s})$.\\
  Structural induction (\myit{構造的帰納法}):\\
  \quad $\forall \texttt{s} \in \mathcal{T},
  (\forall \texttt{r}\in\mathcal{T} ~s.t.~ \text{\texttt{r} is a immediate subterm of \texttt{s}}, P(\texttt{r}) \rightarrow P(\texttt{s}))
  \\\qquad \rightarrow \forall \texttt{s} \in \mathcal{T}, P(\texttt{s})$.\\
\end{theorem}
\end{frame}

\begin{frame}{Proof: Exercise ($\bigstar\bigstar$)}
\pause
Maybe use the concrete definition and induction on natural numbers......
\begin{alignat*}{2}
  \mathcal{S}_0 &= \emptyset \\
  \mathcal{S}_{i+1} &= & &\{ \texttt{true, false, 0} \}\\
  &&\cup &\{ \texttt{succ t\subs{1}, pred t\subs{1}, iszero t\subs{1}} \vert \texttt{t\subs{1}} \in \mathcal{S}_i \}\\
  &&\cup &\{ \texttt{if t\subs{1} then t\subs{2} else t\subs{3}} \vert \texttt{t\subs{1}, t\subs{2}, t\subs{3}} \in \mathcal{S}_i \}\\
\end{alignat*}
\vspace*{-3em}
\begin{align*}
  &\text{let} ~\mathcal{P}_i = (\forall \texttt{s}\in\mathcal{S}_{i},
    (\forall \texttt{r}\in\mathcal{S}_{i} ~s.t.~ depth(\texttt{r}) < depth(\texttt{s}), P(\texttt{r}) \rightarrow P(\texttt{s}))\\
  &\qquad\qquad\rightarrow \forall \texttt{s}\in\mathcal{S}_i, P(\texttt{s})).\\
  &\mathcal{P}_0 \land \left(\forall i \in \mathbb{N}, \mathcal{P}_{i} \rightarrow \mathcal{P}_{i+1}\right) \rightarrow \forall i \in \mathbb{N}, \mathcal{P}_{i}. ~(\text{induction on natural numbers})
\end{align*}
\end{frame}

\begin{frame}{Power of Structural Induction}
Structural induction is often easier than other inductions.
\begin{proof}[Structural Induction]
By induction on \texttt{t}.\\
Case: \texttt{t = true}\\
\dots show $P(\texttt{true})$ \dots\\
Case: \texttt{t = succ t\subs{1}}\\
\dots show $P(\texttt{succ t\subs{1}})$ using $P(\texttt{t\subs{1}})$ \dots\\
Case: \texttt{t = if t\subs{1} then t\subs{2} else t\subs{3}}\\
\dots show $P(\texttt{if t\subs{1} then t\subs{2} else t\subs{3}})$ using $P(t\subs{1}), P(t\subs{2}), P(t\subs{3})$ \dots\\
\end{proof}
\end{frame}

\subsection*{3.4 Semantic Styles}

\begin{frame}{In the First Place...}
Two elements that characterize programming languages:
\begin{itemize}
  \item Syntax (構文)
  \item Semantics (意味論)
\end{itemize}
\end{frame}

\begin{frame}{Three Semantic Styles}
\begin{itemize}
  \item Operational semantics (操作的意味論)
  \item Denotational semantics (表示的意味論)
  \item Axiomatic semantics (公理的意味論)
\end{itemize}
\end{frame}

\begin{frame}{Operational Semantics}
Define an \textit{abstract machine} and specify the behavior.
\begin{block}{Example}
\begin{mathpar}
\texttt{if true then t\subs{2} else t\subs{3}} \Rightarrow \texttt{t\subs{2}} \and
\texttt{if false then t\subs{2} else t\subs{3}} \Rightarrow \texttt{t\subs{3}} \and
\infer{\texttt{if t\subs{1} then t\subs{2} else t\subs{3}} \Rightarrow \texttt{if t${}_1^\prime$ then t\subs{2} else t\subs{3}}}{\texttt{t\subs{1}} \rightarrow \texttt{t}_1^\prime}
\end{mathpar}
\end{block}
We usually use this!
\end{frame}

\begin{frame}{Denotational Semantics}
The meaning of a term is taken to be some mathematical object.
\begin{block}{Example}
\begin{mathpar}
\llbracket 0 \rrbracket = 0 \and \llbracket \texttt{succ t} \rrbracket = \llbracket t \rrbracket + 1 \and \llbracket \texttt{pred t} \rrbracket = \llbracket t \rrbracket - 1
\end{mathpar}
\end{block}
\end{frame}

\begin{frame}{Axiomatic Semantics}
One concrete example is Hoare logic.
\begin{block}{Hoare Triple}
$$\{P\}~C~\{Q\}$$ where
\begin{itemize}
  \item $P$ is a precondition
  \item $C$ is a program
  \item $Q$ is a postcondition
\end{itemize}
If $P$ holds and $C$ executes, then $Q$ holds.\\
Ex. $\{x=1, y=1\}~z:=x+y~\{x=1, y=1, z=2\}$
\end{block}
\end{frame}

\end{document}